\documentclass{book}
\usepackage[utf8]{inputenc}
\usepackage[brazil]{babel}
\usepackage{indentfirst}
\setlength{\voffset}{0pt}
\usepackage[bottom=2cm,top=2cm,left=2cm,right=2cm]{geometry}
\usepackage{amsmath,amssymb,amsthm}
\usepackage{gensymb}
\usepackage{graphicx}
\PassOptionsToPackage{hyphens}{url}\usepackage{hyperref}
\usepackage{url}
\setlength{\parskip}{0.5em}

\title{Inline}
\author{Lucas Seiki Oshiro -- 9298228\\
  Marcos Vinicius do Carmo Sousa -- 9298274\\
  Matheus Tavares Bernardino -- 9292987\\}

\begin{document}
\maketitle
\tableofcontents

\chapter{Introdução}
O sistema InLine se propõe a facilitar o processo de gerenciamento de filas em
estabelecimentos. Em vez de filas presenciais, geralmente gerenciadas por
tickets em papel ou pagers vibratórios, o sistema traz filas virtualizadas, onde
os clientes podem acompanhar sua posição em tempo real sem a necessidade de
estar fisicamente no estabelecimento desejado.

\section{Funcionalidades e Características Principais}
O sistema tem por finalidade principal a execução das seguintes funcionalidades
e características:

\begin{itemize}
\item Implementação de filas por ‘ordem de chegada’ e por ‘hora marcada’.
\item Filas por ‘ordem de chegada’ representam as filas tradicionais, onde os
  usuários entram na última posição e são chamados a partir da primeira. A
  ordenação da fila, porém, além da ordem de chegada, leva em conta uma
  combinação de atributos do cliente, de modo a tornar a fila mais inteligente.
\item Filas de ‘hora marcada’, constituem-se de uma sequência de usuários com
  ordem pré-agendada para atendimento no estabelecimento (como um agendamento
  tradicional em um consultório médico, por exemplo). A diferença nesta
  categoria é que o usuário é atualizado em tempo real sobre seu agendamento. Se
  o estabelecimento, por alterações no tempo previsto de atendimento dos
  clientes anteriores, precisar atrasar o agendamento de um usuário (ou existe
  possibilidade de antecipar), o usuário é notificado.
\item Promover filas com ‘acesso prioritário’ para idosos com idade superior a
  60 anos, aposentados, gestantes, pessoas com crianças de colo, deficientes e
  obesos.
\item Permitir o ‘acesso premium’ para usuários que desejarem pagar uma taxa
  mensal ou comprarem créditos. No primeiro caso, os usuários têm ‘acesso
  premium’ a todas as filas com suporte a premium. No segundo, os usuários podem
  comprar créditos e gastá-los apenas nas filas com suporte a premium que
  desejarem (cada crédito dá direito a um acesso a uma fila). Cabe a cada
  estabelecimento decidir se suas filas terão suporte a ‘acesso premium’ ou não.
\item Para clientes que chegarem a um estabelecimento sem possuir o app InLine
  ou acesso à internet, o sistema permitirá sua entrada na fila como um ‘usuário
  não-cadastrado’. Sua entrada será feita pelo gerente do estabelecimento com o
  nome do cliente.
\item Mostrar aos usuários, estabelecimentos próximos a ele e estabelecimentos
  onde as filas estão pequenas.
\item Cada usuário na fila do tipo ‘ordem de chegada’ tem sua posição GPS
  monitorada de tempos em tempos, enquanto estiver na fila. Caso sua posição GPS
  não se aproxime do estabelecimento conforme sua vez for chegando na fila,
  ocorre uma penalização de posições.
\end{itemize}

\section{Aplicações}
O Inline pode ser utilizado em dois tipos de aplicações. Uma delas é uma
interface web destinada às empresas e estabelecimentos, e outra é um PWA
destinado aos clientes dos estabelecimentos.

A aplicação web destinada a empresas e estabelecimentos é uma interface para a
criação e gerenciamento das filas.

O PWA (Progressive Web App) é uma tecnologia que permite que um site funcione
como um aplicativo nativo, podendo ser instalado no celular. Através do PWA
do Inline o cliente interage com fila e com os estabelecimentos.

\chapter{Instalação}

\section{Servidor}
\subsection{Banco de Dados PostgreSQL}
\subsection{Banco de Dados MongoDB}
\subsection{API}
\subsection{Front-end (React)}
\subsection{Front-end (PWA)}

\section{Cliente}
\subsection{Site (empresas e estabelecimentos)}
\subsection{PWA (clientes)}
Ao abrir o PWA pelo navegador Chrome em um dispositivo Android, basta ir no
menu, e clicar em ``Adicionar à tela inicial''. Aparecerá o ícone do Inline
na tela inicial do Android do utilizador, que leva ao PWA instalado.

Contudo, apesar de facilitar o seu uso, não é necessário fazer esse procedimento
para utilização do PWA, podendo ser utilizado diretamente pelo navegador, caso o
usuário prefira.

\chapter{Utilização do Site}

\chapter{Utilização do PWA}

O PWA do Inline pode ser encontrado em \url{http://35.231.149.80:8080/}. Ao
entrar nele, é apresentada a tela de \textit{login}. 

\section{Login}

São pedidos o e-mail e a senha de um cliente. Uma vez fornecidos, basta clicar
no botão \textit{LOGIN}, que ele estará logado no sistema. Uma vez logado, ele
irá para a tela das empresas disponíveis.

\section{Cadastro}
São pedidos o nome, e-mail, senha, celular e o tipo de prioridade do usuário,
caso possua. Após clicar em \textit{CRIAR}, o usuário estará cadastrado no
sistema caso os dados fornecidos estiverem corrretos. Após isso, será carregada
a tela de \textit{login}.

\section{Empresas disponíveis}
São listadas as empresas que estão disponíveis no sistema Inline. Ao clicar em
uma delas, será carregada a lista de estabelecimentos dela que estão disponíveis
no sistema.

\section{Estabelecimentos disponíveis}
São listados os estabelecimentos disponíveis da empresa escolhida, e que estão
cadastrados no sistema Inline. Ao clicar em um estabelecimento, será carregada a
tela que contém informações sobre ele.

\section{Estabelecimento}
São mostradas informações sobre o estabelecimento, as filas disponíveis nele, e
avaliações de usuários. Ao clicar em uma fila, o usuário é redirecionado pra a
tela da fila.

\section{Fila}
Caso o usuário já esteja na fila, é mostrada a posição em que ele está. Caso não
esteja, o usuário pode se inscrever nela.

Em qualquer momento, o usuário pode voltar para a tela anterior utilizando o
botão no canto superior esquerdo.

A qualquer momento em que o usuário estiver logado, ele pode se deslogar a
partir do menu lateral.

\end{document}
